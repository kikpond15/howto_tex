% 独自のコマンド

% ■ アブストラクト
%  \begin{jabstract} 〜 \end{jabstract}  :日本語のアブストラクト
%  \begin{eabstract} 〜 \end{eabstract}  :英語のアブストラクト

% ■ 謝辞
%  \begin{acknowledgment} 〜 \end{acknowledgment}

% ■ 文献リスト
%  \begin{bib}[100] 〜 \end{bib}

\newif
\ifjapanese
\japanesetrue  % 論文全体を日本語で書く(英語で書くならコメントアウト)

\ifjapanese
  %\documentclass[a4j,twoside,openright,11pt]{jreport} % 両面印刷の場合。余白を綴じ側に作って右起こし。
  \documentclass[a4j,11pt]{jreport}  % 片面印刷の場合。
  \renewcommand{\bibname}{参考文献}
  \newcommand{\acknowledgmentname}{謝辞}
\else
  \documentclass[a4paper,11pt]{report}
  \newcommand{\acknowledgmentname}{Acknowledgment}
\fi
\usepackage{thesis}
\usepackage{ascmac}
%\usepackage{graphicx}
\usepackage[dvipdfmx]{graphicx}
\usepackage{multirow}
\usepackage{url}
\bibliographystyle{junsrt}
%\bibliographystyle{jplain}
\bindermode  % バインダー用余白設定

% 日本語情報(必要なら)
\jclass  {卒業論文}                             % 論文種別
\jtitle    {
 TeXで書く : 卒論フォーマットを用いた\\ \ TeXの書き方についての説明
}    % タイトル。改行する場合は\\を入れる
%\jtitle    {KUI : 影ユーザーインタフェース\\影をインタフェースとしたインタラクション手法の提案}  
\juniv    {明星大学}                  % 大学名
\jfaculty  {情報学部 情報学科\\ 尼岡研究室}               % 学部、学科
\jauthor  {13J5-048\\菊池 康太}                       % 著者
\jhyear  {1}                                   % 平成○年度
\jsyear  {2019}                                 % 西暦○年度
\jkeyword  {TeX,LaTeX,卒論}     % 論文のキーワード
\jproject{}%インタラクションデザインプロジェクト} %プロジェクト名
\jdate{令和1年度}

% 英語情報(必要なら)
\eclass  {Master's Thesis}                            % 論文種別
\etitle    {A \LaTeX Template for Master Thesis}      % タイトル。改行する場合は\\を入れる
\euniv  {Meisei University}                             % 大学名
\efaculty  {Graduate School of Media and Governance}  % 学部、学科
\eauthor  {Fusuke Hogeyama}                           % 著者
\eyear  {2015}                                        % 西暦○年度
\ekeyword  {\LaTeX, Templete, Master Thesis}          % 論文のキーワード
\eproject{Interaction Design Project}                 %プロジェクト名
\edate{January 2015}



\begin{document}
%\ifjapanese
 \jmaketitle    % 表紙(日本語)
%\else
%  \emaketitle    % 表紙(英語)
%\fi

% ■ アブストラクトの出力 ■
%	◆書式:
%		begin{jabstract}〜end{jabstract}	:日本語のアブストラクト
%		begin{eabstract}〜end{eabstract}	:英語のアブストラクト
%		※ 不要ならばコマンドごと消せば出力されない。



% 日本語のアブストラクト
\begin{jabstract}

本稿では,卒業論文をTexで書こうとしている,もしくはTeXで書けと脅されている人へ向けた卒論フォーマットである.
タイトルや著者名.論文内の文章をそのまま読者のものに置き換えることで,体裁の整った卒業論文を作成することが可能である.
Wordでの卒論執筆は文章が長くればなるほど編集が困難になるが,Texを使えば図の配置や参考文献の参照など自動で
変更してくれるため,一度Texの書き方に慣れさえすれば容易に綺麗な卒業論文を作成することができる.

また,卒業論文執筆とは一生に一度,経験するかしないかの体験であり,
大学生活4年間の結晶でもある.
文章の内容が良くても体裁の取れていない論文では,読む人へ与える印象は大きく変わってくる.
どうせなら良い内容,良い体裁で納得のいく卒業論文を作ろうではないか.


\end{jabstract}

  % アブストラクト。要独自コマンド、include先参照のこと

\tableofcontents  % 目次
\listoffigures    % 表目次
%\listoftables    % 図目次

\pagenumbering{arabic}

\chapter{はじめに}
\label{chap:introduction}

この章では,おもに研究に至った背景について述べる.


研究が問題解決を目的としているなら,その社会的課題について社会的背景について述べると良い.
研究が問題定義を目的,実験的な研究であれば,なぜその定義を思うに至ったのかなど,
著者の経験や思想について述べる.


  % 本文1,01.texを編集しタイプセットはmain.texで行う!!
\chapter{関連研究}
\label{research}

関連研究について述べる.
関連研究は多い方が良く,複数の視点から比較できた方がよい.

また,関連研究との差異を述べることで,本研究の新規性を示すこともできる.








  % 本文2
\chapter{目的}
\label{chap:goal}
本章では,本研究の目的及び新規性について述べる.
また,本研究によってもたらされる,期待される結果についても述べる.
  % 本文3
\chapter{本研究について}
\label{chap:kui}

この章は目的の章と一緒にしても良い.
研究の概念や,新規性,特徴など特出する点を述べる.







  % 本文4
\chapter{提案システムの設計}
システムの設計について,なぜその設計方法にしたのか,目的に沿った設計になっているかなど踏まえて書けると良い.

ソフトウェア単体の場合はソフトウェア設計だけでも良い.
その代わり,UIやシステムの処理工程などフローチャートを書くと良い.

\section{ハードウェア設計}
制作物がある場合(装置がある場合)は,大まかな設計図を入れると良い.


\section{ソフトウェア設計}
  % 本文5
\chapter{提案システムの実装}
\label{chap:implementation}

\section{開発環境・機材}
設計に基づき実際に使用する開発環境や
機材について説明する.
なぜその機材を選定したかなど理由を書いても良い.

\section{実装方法}
具体的な実装方法.
処理の手順や物がある場合は実寸の大きさなど記載する.



\section{動作の様子}
実際に動作させている様子を図を交えて述べる.
どうさパターンや体験の様子などいれられると良い.
また,動作速度が影響を与えそうな研究の場合は,動作速度や機能的分解能,追従性なども必要に応じて記載する.

  % 本文6
\chapter{評価実験}
\label{chap:evaluate}

\section{実験概要}
何を目的に実験を行うか,この実験でどんな事を図りたいかなど実験の趣旨を説明する.
また,実験を行う環境,機材,日時,実験対象などの詳細も記載する.

\section{評価項目}
評価実験で評価する項目の説明を記載する.実験の目的からなぜこの項目を設けたのかまで説明できると良い.



\section{実験結果}
実験の結果をグラフや数値で示し,どんな結果になったか事実を述べる.
ここで実験結果の考察をしても良い.次の章でまとめて考察しても良い.


  % 本文7
\chapter{考察}
この章では主に,評価実験から得られた結果,本研究の目的は達成できたかなど考察を述べる.
目的に合わせた設計手法や提案手法が適切だったかなど.実験結果の数値から考察できると良い.


\section{議論}
実験結果の考察から,今後システムには何が必要かなど議論する


\section{課題}
データから見られる今後の課題や,実験で発生した問題など解決方法を踏まえて書けると良い.


\section{今後の展望}
本研究を発展させたら,他にどんなことができるか,他の利用シーンではこんな使い方も想定
できるなど,この研究がこれで終わりではく,もっと可能性を秘めていることを展望に乗せて書けると良い.





  % 本文8
\chapter{終りに}
\label{chap:conclusion}
ここでは,論文全体のまとめを書きます.
提案したこと,作成したシステム,評価実験から得られた結果,課題や展望,
そしてこの研究が今後どうなっていくかや,どんな場面で使用されるかなど期待を書くと良い.
  % 本文9
\begin{acknowledgment}
ここでは謝辞を書く.
指導教員や相談に乗ってくれた他教員,先輩,後輩,研究室の同期など,
研究に携わってくれた感謝を伝える.

\end{acknowledgment}
  % 謝辞。要独自コマンド、include先参照のこと
\begin{bib}[100]
  \bibliography{main}
\end{bib}
  % 参考文献。要独自コマンド、include先参照のこと(main.bib を編集する)
%\appendix
%\chapter{付録の例}

付録を無理矢理出力させるため、てきとうなことを書く。

\section{ほげ}

コマンドは本文と一緒。

\subsection{ふー}

本文と一緒。

\section{ほげほげ}

本文と一緒。

\subsection{ふーふー}

本文と一緒。
    % 付録

\end{document}
